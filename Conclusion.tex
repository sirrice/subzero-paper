\section{Conclusion}

This paper introduced \sys{}, a scientific-oriented lineage storage and query
system that stores a mix of black-box and fine-grained lineage.  \sys{} uses an
optimization framework that picks the lineage representation on a per-operator
basis that maximizes lineage query performance while staying within user
constraints.  In addition, we presented {\it region lineage}, which explicitly
represents lineage relationships between sets of input and output data
elements, along with a number of efficient encoding schemes.  \sys{} is heavily
optimized for operators that can deterministically compute lineage from array
cell coordinates and small amounts of operator-generated metadata.  UDF
developers expose  lineage relationships and semantics by calling the runtime
API and/or implementing mapping functions.

Our experiments show that many scientific operators can use our techniques to
dramatically reduce the amount of redundant lineage that is generated and
stored to  improve query performance by up to 10$\times$ while using up to
70$\times$ less storage space as compared to existing cell-based strategies.
The optimizer successfully scales the amount of lineage stored based on
application constraints, and can improve the query performance of the genomics
benchmark, which is amenable to black-box only strategies..  In conclusion,
\sys{} is an important initial step to make interactively querying 
fine-grained lineage a reality for scientific applications.


%to this general framework,
%we introduced {\it region provenance}, which encodes provenance between sets of
%input and output data elements and efficiently captures the locality that many
%scientific operators exhibit (e.g., a star in LSST is detecting by looking a
%clusters of adjacent pixels).  We proposed several efficient representations of
%{\it region provenance} and encoding schemes that trade off storage size and
%query performance.  \sys{} additionally provides developers an API to write
%provenance from within user-defined operators.

%Our experiments were based on two real-world benchmarks.  The astronomy
%benchmark reflects common scientific workflows and shows that our techniques
%can improve query performance by up to 10$\times$ while requiring 70$\times$
%less storage space as compared to traditional cell-provenance strategies.  The
%second genomics benchmark executes the workflow very quickly and generates lots
%of provenance.  Even though it is more amenable to black-box only strategies,
%\sys{} is able to optimally use the available storage space and still improve
%provenance query performance.  

