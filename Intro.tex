\section{Introduction}

Many scientific applications are naturally expressed as a workflow that
comprises a sequence of operations applied to raw input data to produce an
output dataset or visualization.  Like database queries, such workflows can be
quite complex, consisting up to hundreds of operations~\cite{lsst} whose
parameters or inputs vary from one run to another.  

Scientists record and query provenance -- metadata that describes the
processes, environment and relationships between input and output data arrays
-- to ascertain data quality, audit and debug workflows, and more generally
understand how the output data came to be. A key component of provenance, {\it
data lineage}, identifies how input data elements are related to output data
elements and is integral to debugging workflows.  For example, scientists need
to be able to work backward from the output to identify the sources of an error
given erroneous or suspicious output results.  Once the source of the error is
identified, the scientist will then often want to identify derived downstream
data elements that depend on the erroneous value so he can inspect and possibly
correct those outputs.  



In this paper, we describe the design of a fine-grained lineage tracking and
querying system for array-oriented scientific workflows.  We assume a data and
execution model similar to SciDB~\cite{scidb}.  We chose this because it
provides a closed execution environment  that can capture all of the lineage
information, and because it is specifically designed for scientific data
processing (scientists typically use RDBMSes to manage metadata and do data
processing outside of the database).  The system allows scientists to perform
exploratory workflow debugging by executing a series of {\it data lineage
queries} that walk backward to identify the specific cells in the input arrays
on which a given output cell depends and that walk forward to find the output
cells that a particular input cell influenced.  Such a system must manage input
to output relationships at a {\it fine-grained} array-cell level.

Prior work in data lineage tracking systems has largely been limited to
coarse-grained metadata tracking~\cite{taverna,genepattern}, which stores
relationships at the file or relational table level. {\it Fine-grained lineage}
tracks relationships at the array cell or tuple level.  The typical approach,
popularized by Trio~\cite{trio}, which we call {\it cell-level lineage},
eagerly materializes the identifiers of the input data records (e.g., tuples or
array cells) that each output record depends on, and  uses it to
directly answer backward lineage queries. An alternative, which we call {\it
black-box lineage}, simply records the input and output datasets and runtime
parameters of each operator as it is executed, and materializes the lineage at
lineage query time by re-running relevant operators in a tracing mode.

Unfortunately, both techniques are insufficient in scientific
applications for two reasons. First, scientific applications make heavy use of
user defined functions (UDFs), whose semantics are opaque to the lineage
system.  Existing approaches  conservatively assume that every output cell of a
UDF depends on every input cell, which limits the utility of a fine-grained
lineage system because it tracks a large amount of information without
providing any insight into which inputs actually contributed to a given output.
This necessitates proper APIs so that UDF designers can expose fine-grained
lineage information and operator semantics to the lineage system.

Second, neither black-box only nor cell-level only techniques are sufficient
for many applications. Scientific workflows consume data arrays that regularly
contain millions of cells, while generating complex relationships between
groups of input and output cells.    Storing cell-level lineage can avoid
re-running some computationally intensive operators (e.g., an image processing
operator that detects a small number of stars in telescope imagery), but needs
enormous amounts of storage if every output depends on every input (e.g., a
matrix sum operation) -- it may be preferable to recompute the lineage at
query time.  In addition, applications such as LSST\footnote{http://lsst.org}
are often subject to  limitations that only allow them to dedicate a
small percentage of storage to lineage operations.   Ideally, lineage 
systems would support a hybrid of the two approaches and take user constraints
into account when deciding which operators to store lineage for.


This paper seeks to address both challenges. We interviewed scientists from
several domains to understand their data processing workflows and  lineage
needs and used the results to design a science-oriented data lineage system.
We introduce {\it Region Lineage}, which exploits locality properties prevalent
in the scientific operators we encountered.  It addresses common relationships
between regions of input and output cells by storing grouped or summary
information rather than individual pairs of input and output cells. We
developed a lineage API that supports black-box lineage as well as {\it Region
Lineage}, which subsumes cell-level lineage.  Programmers can also specify
forward/backward {\it Mapping Functions} for an operator to directly
compute the forward/backward lineage solely from input/output cell coordinates
and operator arguments; we implemented these for many common matrix and
statistical functions.  We also developed a hybrid lineage storage
system that allows users to explicitly trade-off storage space for lineage
query performance using an optimization framework.  Finally, we introduce two
end-to-end scientific lineage benchmarks.

As mentioned earlier, the system prototype, \sys{}, is implemented in the
context of the SciDB model.  SciDB stores multi-dimensional arrays and executes
database queries composed of built-in and user-defined operators (UDFs) that
are compiled into workflows.  Given a set of user-specified storage
constraints, \sys{} uses an optimization framework to choose the optimal type
of lineage (black box, or one of several new types we propose) for each SciDB
operator that minimizes lineage query costs while respecting user storage
constraints.


A summary of our contributions include: 

\begin{CompactEnumerate}

\item The notion of {\it region lineage}, which \sys{} uses to efficiently
    store and query lineage data from scientific applications.  We also
    introduce several efficient representations and encoding schemes that each
    have different overhead and query performance trade offs.

\item A {\it lineage API} that operator developers can use to expose lineage
    from user defined operators, including the specification of {\it mapping
    functions} for many of the built in SciDB operators.

\item A {\it unified storage model} for mapping functions, region and
    cell-level lineage, and black-box lineage.

\item An {\it optimization framework} which picks an optimal mixture of
    black-box and region lineage  to maximize query
    performance within user defined constraints.

\item A performance evaluation of our approach on end-to-end astronomy and
    genomics benchmarks.  The astronomy benchmark, which is computationally
    intensive but exhibits high locality, benefits from efficient
    representations.  Compared to cell-level and black-box lineage, \sys{}
    reduces storage overhead by nearly 70$\times$ and speeds query performance
    by almost 255$\times$.  The genomics benchmark highlights the need for, and
    benefits of, using an optimizer to pick the storage layout, which improves
    query performance by 2--3$\times$ while staying within user constraints.
    

\end{CompactEnumerate}

The next section  describes our motivating use cases in more detail.  It is
followed by a high level system architecture and details of the rest of the
system.













