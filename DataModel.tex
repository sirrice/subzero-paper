\section{Data, Lineage and Query Model}
\label{s:datamodel}

In this section, we describe the representation and notation of lineage 
data and queries in \sys{}.

\sys{} is designed to work with a workflow executor system that applies a fixed
sequence of operators to some set of inputs.  Each operator operates
on one or more input objects (e.g., tables or arrays), and produces a
single output object.  Formally, we say an operator $P$ takes as input
$n$ objects, $I_P^1,...,I_P^n$, and outputs a single object, $O_P$.

Multiple operators are composed together to form a workflow, described by a
workflow specification, which is a directed acyclic graph $W = (N, E)$, where
$N$ is the set of operators, and $e = (O^P, I^{P'}_i) \in E$ specifies that the
output of $P$ forms the {\it i'th} input to the operator $P'$.  An instance of
$W$, $W_j$, executes the workflow on a specific dataset. Each operator runs
when all of its inputs are available.

The data follows the SciDB data model, which processes multi-dimensional
arrays.  A combination of values along each dimension, termed a {\it
coordinate}, uniquely identifies a cell.  Each cell in an array has the same
schema, and consists of one or more named, typed fields.  SciDB is ``no
overwrite,'' meaning that intermediate results produced as the output of an
operator are always stored persistently, and each update to an object creates a
new, persistent version.  \sys{} stores lineage information
with each version to speed up lineage queries. 


Our notion of backward lineage is defined as a subset of the inputs that will
reproduce the same output value  if the operator is re-run on its lineage.  For
example, the lineage of an output cell of Matrix Multiply are all cells of the
corresponding row and column in the input arrays -- even if some are empty.
Forward lineage is defined as a subset, $C$, of the outputs such that the
backward lineage of $C$ contains the input cells.  The exact semantics for UDFs
are ulitmately controlled by the developer.

%This definition is consistent with other provenance models that assume
%all-to-all relationships across UDFs.  While this is sufficient for the
%applications we encountered, enforcing stricter semantics in lineage systems
%that support UDFs is an important subject of future work.




%Finally, we define the $fanout$($fanin$) of an operator as the average number
%of output(input) cells that are derived from each input(output) cell. 

%Black-box lineage only stores references to the input and output arrays as
%each operator is executed.  
\sys{} supports three types of lineage: {\it black box},  {\it cell-level}, and {\it region}
lineage.
As a workflow executes, lineage is generated on an operator-by-operator basis,
depending on the types of lineage that each operator is instrumented to support
and the materialization decisions made by the optimizer.  We have instrumented
SciDB's built-in operators to generate lineage mappings from inputs to
outputs and provide an API for
UDF designers to expose these relationships.  If the API is not used, then \sys{}
assumes an all-to-all relationship between the cells of the input arrays and
cells of the output array.



\paragraph{Black-box lineage}\sys{} does not require additional resources to store black-box lineage because, like SciDB,
our workflow executor records intermediate results as well as input and output array versions
as peristent, named objects. These are sufficient to re-run any previously executed
operator from any point in the workflow.

%Black box lineage allows any output cell to be recreated by re-running the
%operator with the original inputs. % and filtering the lineage that is
%generated.

\paragraph{Cell-level lineage} Cell-level lineage models the relationships
between an output cell and each input cell that generated it \footnote{Although
    we model and refer to lineage as a mapping between input and output cells,
    in the \sys{} implementation we store these mappings as references to
physical cell coordinates.} as a set of pairs of input and output cells:
{\footnotesize $$\{(out,in) | out \in O_P \wedge in \in \cup_{i\in [1,n]} I_P^i
\}$$ } Here, $out \in O_P$ means that $out$ is a single cell contained in the
output array $O_P$.  $in$ refers to a single cell in one of the input arrays.

\paragraph{Region lineage} Region lineage 
models lineage as a set of {\it region pairs}.  Each region pair describes an
all-to-all lineage relationship between a set of output cells, $outcells$, and
a set of input cells, $incells_i$, in each input array, $I_P^i$: 
{\footnotesize
    $$\{(outcells,incells_1,...,incells_n) | outcells \subseteq O_P
    \wedge incells_i \subseteq I_P^i\}$$ }
Region lineage is more than a short hand; scientific applications often exhibit
locality and generate multiple output cells from the same set of input cells,
which can be represented by a single region pair.  For example, the LSST star
detection operator finds clusters of adjacent bright pixels and generates an
array that labels each pixel with the star that it belongs to.  Every output
pixel labeled {\it Star X} depends on all of the input pixels in the {\it Star
X} region.  Automatically tracking such relationships at the cell level  is
particularly expensive, so region lineage is a generalization of cell-level
lineage that makes this relationship explicit.  For this reason, later sections
will exclusively discuss region pairs.




%\subsection{Query Model}

Users execute a lineage query by specifying the coordinates of
an initial set of query cells, $C$, in a starting array, and a path of
operators $(P_1 \ldots P_m)$  to trace through the workflow:
{\footnotesize
    $$R = execute\_query(C, ((P_1, idx_1),..., (P_m, idx_m)))$$
}
Here, the indexes ($idx_1 \ldots idx_m$) are used to disambiguate which input
of a multi-input operator that the query path traverses.  

Depending on the order of operators in the query path, \sys{} recognizes the
query as a {\it forward lineage query} or {\it backward lineage query}.  A {\it
forward lineage query} defines a path from some ancestor operator $P_1$ to some
descendent operator $P_m$.  The output of an operator $P_{i-1}$ is  the
$idx_i$'th input of the next operator, $P_i$.  The query cells $C$ are a
subset of $P_1$'s $idx_1$'th input array, $C \subseteq I_{P_1}^{idx_1}$.

A {\it backward lineage query} reverses this process, defining a path from some
descendent operator, $P_1$ that terminates at some ancestor operator, $P_m$.
The output of an operator, $P_{i+1}$ is the $idx_i$'th input of the previous
operator, $P_i$, and the query cells $C$ are a subset of $P_1$'s output array,
$C \subseteq O_{P_1}$.  
The query results are the coordinates of the cells $R \subseteq O_{P_m}$ or $R
\subseteq I_{P_m}^{idx_m}$, for forward and backward queries, respectively.



%It is also used when the output of an operator is used
%as multiple inputs to
%the same operator (e.g., self-join).  








