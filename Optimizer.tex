
\section{Lineage Strategy Optimizer}
\label{s:optimizer}

Having described the basic storage strategies implemented in \sys{},
we now describe our lineage storage optimizer.
The optimizer's objective is to choose a set of {\it storage strategies} that
minimize the cost of executing the workflow while keeping storage overhead
within user-defined constraints.  We formulate the task as an integer
programming problem, where the inputs are a list of operators, strategy pairs,
disk overheads, query cost estimates, and a sample workload that is used to
derive the frequency with which each operator is invoked in the lineage
workload.  Additionally, users can manually specify operator specific
strategies prior to running the optimizer. 


The formal problem description is stated as:

{\footnotesize
\[
\begin{array}{l l l}
\min_x          & \sum_i p_i * \left(\min_{j|x_{ij}=1} q_{ij}\right) +
&\epsilon * \sum_{ij} (disk_{ij} + \beta * run_{ij}) * x_{ij} \\
%                &\epsilon * \sum_{ij} disk_{ij} * x_{ij} \\
\mbox{s.t.}   &\sum_{ij} disk_{ij} * x_{ij} &\leq MaxDISK\\
              &\sum_{ij} run_{ij} * x_{ij} &\leq MaxRUNTIME\\
              &\forall_{i} \left(\sum_{0 \leq j < M} x_{ij}\right) &\ge 1\\
              &\forall_{i,j} x_{ij} &\in \{0, 1\} \\
&&\\
              &\hspace*{-.3in}  {\rm user\ specified\ strategies}&\\
              &x_{ij} = 1 &\forall_{i,j} x_{ij} \in U
\end{array}
\]
}

Here, $x_{ij} = 1$ if operator $i$ stores lineage using strategy $j$, and 0
otherwise.  $MaxDISK$ is the maximum storage overhead specified by the user;
$q_{ij}$, $run_{ij}$, and $disk_{ij}$, are the average query cost, runtime
overhead, and storage overhead costs for operator $i$ using strategy $j$ as
computed by the cost model.  $p_{ij}$ is the probability that a lineage
query in the workload accesses operator $i$, and is computed from the sample
workload.  A single operator may store its lineage data using multiple
strategies.

The goal of the objective function is to minimize the cost of executing the
lineage workload, preferring strategies that use less storage.  When an
operator uses multiple strategies to store its lineage, the query processor
picks the strategy that minimizes the query cost.  The $\min$ statement in the
left hand term picks the best query performance from the strategies that have
been picked ($j | x_{ij} = 1$).  The right hand term penalizes strategies that
take excessive disk space or cause runtime slowdown.  $\beta$ weights runtime
against disk overhead, and $\epsilon$ is set to a very small value to break
ties. A large $\epsilon$ is similar to reducing $MaxDISK$ or $MaxRUNTIME$.

We heuristically remove configurations that are clearly non-optimal, such as
strategies that exceed user constraints, or are not properly indexed for any of
the queries in the workload (e.g., forward optimized when the workload only
contains backward queries).  The optimizer also picks mapping functions over
all other classes of lineage.

We solve the ILP problem using the simplex method in GNU Linear Programming
Kit.  The solver takes about 1ms to solve the problem for the 
benchmarks.




\subsection{Query-time Optimizer}
\label{s:cmopt}

While the lineage strategy optimizer picks the optimal lineage strategy, the
executor must still pick between accessing the lineage stored by one of the
lineage strategies, or re-running the operator.  The query-time optimizer
consults the cost model using statistics gathered during query execution and
the size of the query result so far to pick the best execution method.  In
addition, the optimizer monitors the time to access the materialized lineage.
If it exceeds the cost of re-executing the operator, \sys{} dynamically
switches to re-running the operator.  This bounds the worst case performance to
2$\times$ the black-box approach.






