\section{Related Work}


%Many of our ideas are
%inspired by previous work, which we highlight in this section.


There is a long history of provenance and lineage research both in database
systems and in more general workflow systems.   There are several excellent
surveys that characterize provenance in databases~\cite{provdb} and scientific
workflows~\cite{provsci,provsci2}.  As noted in the introduction, the primary
differences from prior work  are that \sys{} uses a mix of black-box and region
provenance, exploits the semantics of scientific operators (making using of
mapping functions) and uses a number of provenance encodings.  

Most workflow systems support custom operators containing user-designed code
that is opaque to the runtime.  This presents a difficulty when trying to
manage cell-level (e.g., array cells or database tuples) provenance.  Some
systems~\cite{genepattern,galaxy} model operators as black-boxes where all
outputs depend on all inputs, and track the dependencies between input and
output datasets.  Efficient methods to expose, store and query cell-level
provenance is an area of on-going research.

Several projects exploit workflow systems that use high level programming
constructs with well defined semantics.  RAMP~\cite{ramp} extends MapReduce to
automatically generate lineage capturing wrappers around Map and Reduce
operators.  Similarly, Amsterdamer et al~\cite{lipstick} instrument the
PIG~\cite{pig} framework to track the lineage of PIG operators.  However,
 user defined operators are treated as black-boxes, which limits their ability to
track lineage. 

Other workflow systems (e.g., Taverna~\cite{taverna} and Kepler~\cite{kepler}),
process nested collections of data, where  data items may be  imagees or DNA
sequences.  Operators  process  data items in a collection, and these systems
automatically track which subsets of  the collections were modified, added, or
removed~\cite{keplerfine,tavernafine}.  Chapman et. al~\cite{chapman} attach
to each data item a provenance tree of the transformations resulting in the
data item, and propose efficient compression methods to reduce the tree size.
However, these systems  model  operators as black-boxes and data items
are typically files, not records.

Database systems execute queries that process  structured tuples using 
 well defined relational operators, and are a natural target for a
lineage system.  Cui et. al~\cite{tracingcui} identified efficient
tracing procedures for a number of operator properties.  These procedures are
then used to execute backward lineage queries.  However, the model does not
allow arbitrary operators to generate lineage, and  models them as
black-boxes.  Section~\ref{s:storagemodel} describes several mechanisms (e.g.,
payload functions) that can implement many of these procedures.

Trio~\cite{trio} was the first database implementation of cell-level
lineage, and unified uncertainty and provenance under a single
data and query model.  Trio explicitly stores relationships between
input and output tuples, and is analogous to the full provenance
approach described in Section \ref{s:storagemodel}.

The \sys{} runtime API is inspired by the PASS~\cite{pass,passv2} provenance
API.  PASS is a file system that automatically stores provenance information of
files and processes.  Applications can use  the {\it libpass} library to create
abstract provenance objects and relationships between them, analagous to
producing cell-level lineage.  \sys{} extends this API to support the
semantics of common scientific provenance relationships. 


% ExSPAN~\cite{} use declarative networking to record and query the provenance
% of network protocols.

\if{0}
\srm{I think you can cut this.}
{\bf  mention theoretical work?  There's just SO MUCH stuff.  No
  wonder there's so many surveys}
There has been a large body of theoretical work. Semirings relates
provenance to Nested Relational Calculus.  Ibis developed a model and
query language for hierarchical provenance.  Trio and Cui formalized
provenance in database systems.  Weakly invertable functions.
~\cite{semirings,ibis,trio,weakinverse}.
\fi

% There is an additional large body of work focused on compressing
% provenance to reduce the storage requirements.  jagadesh, gustavo, rue

% There are a large number of workflow systems that track the provenance
% at the dataset level (e.g., coarse-grained provenance).  These
% systems~\cite{kepler,taverna,genepattern} have been adopted by
% different scientific communities.  Additional systems can be found in
% the survey~\cite{provdb,provsci}.  \srm{Say how we are different -- repeat what is in
% intro.  Are you absolutely sure about this claim???}


% There has been less work on implementing support for the type of
% fine-grained provenance described in this paper.
% Cui~\cite{tracingcui} defined classes of efficient tracing algorithms
% in the context of warehousing ETL pipelines. Their focus was on
% classifying various mapping functions.  HOW RELATE TO SUBZERO




% {\bf SRM lots and lots of work on scientific workflows and provenance
% e.g.:

% \url{http://www.cs.panam.edu/~artem/main/publications/TR-DB-052007-CFLLF.pdf}

% \url{http://daks.ucdavis.edu/~ludaesch/Paper/qlp-ssdbm09.pdf}


% (and many more -- google ``scientific workflow provenance'')
% }


% How provenance systems relate to Materialized Views.

% Debugging, program replay, Retro?

