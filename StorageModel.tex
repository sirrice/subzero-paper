\section{Lineage API and Storage Model}
\label{s:storagemodel}

\sys{} allows developers to write operators that efficiently represent and
store  lineage.  This section describes several modes of region lineage, and an
API that UDF developers can use to generate lineage from within the operators.  We
also introduce a mechanism to control the modes of lineage that an operator
generates.  Finally, we describe how \sys{} re-executes black-box operators
during a lineage query.  Table \ref{t:api} summarizes the API calls and
operator methods that are introduced in this section.

Before describing the different lineage storage methods, we illustrate the
basic structure of an operator:

{\footnotesize
\begin{alltt}
class OpName:
   def run(input-1,...,input-n,cur_modes):
      /* Process the inputs, emit the output */
      /* Record lineage modes specified
         in cur_modes */
    def supported_modes():
      /* Return the lineage modes the
         operator supports */
\end{alltt}
}

Each operator implements a {\it run()} method, which is called when inputs are
available to be processed.  It is passed a list of lineage modes it should
output in the {\it cur\_modes} argument; it writes out lineage data using
the {\it lwrite()} method described below.  The developer specifies the modes
that the operator supports (and that the runtime will consider) by overriding
the {\it supported\_modes()} method.  If the developer does not override {\it
supported\_modes()}, \sys{} assumes an all-to-all relationship between the
inputs and outputs.  Otherwise, the operator automatically supports black-box
lineage.



\begin{table}[t!]
\vspace{-.2in}
\advance\leftskip-.3in
\footnotesize
\caption{Runtime and operator methods}
%\begin{center}
  \begin{tabular}{ | l | l  | }
    \hline
    {\bf API Method} & {\bf Description} \\ \hline \hline
    \multicolumn{2}{|c|}{\bf System API Calls} \\ \hline
    lwrite(outcells, ${\rm incells_1}$, ...,$ {\rm incells_n}$) & API
    to store lineage relationship.   \\ \hline
    lwrite(outcells, payload) & API to store small binary payload\\
                              &instead of input cells.  Called by \\
			      &payload operators.  \\ \hline
    \multicolumn{2}{|c|}{\bf Operator Methods} \\ \hline
    run(input-1,...,input-n,cur\_modes)
                              & Execute the operator, generating \\
			      & lineage types in cur\_modes $\subseteq \{Full,$\\
			      &$Map, Pay, Comp, Blackbox\}$  \\ \hline

    ${\rm map_b}$(outcell, i) & Computes the input cells in ${\rm input_i}$\\
                              &that contribute to ${\rm outcell}$.\\ \hline
    ${\rm map_f}$(incell, i) & Computes the output cells that depend\\
                             &on ${\rm incell} \in {\rm input_i}$. \\ \hline
    ${\rm map_p}$(outcell, payload, i)
                             & Computes the input cells in  ${\rm input_i}$ \\
			     &that contribute to ${\rm outcell}$, has access \\
			     &to ${\rm payload}$. \\ \hline
    supported\_modes()  &  Returns the lineage modes $C \subseteq \{Full,$\\
                        &$Map, Pay, Comp, Blackbox\}$ \\
    &that the operator can generate. \\ \hline


\end{tabular}
%\end{center}
\label{t:api}
\end{table}



For ease of explanation, this section is described in the context of
the LSST operator $CRD$ (cosmic ray detection, depicted as A and B in
Figure~\ref{f:lsstworkflow}) that finds pixels containing cosmic rays
in a single image, and outputs an array of the same size.  If a pixel
contains a cosmic ray, the corresponding cell in the output is set to
$1$, and the output cell depends on the 49 neighboring pixels within a
3 pixel radius.  Otherwise the output cell is set to $0$, and only
depends on the corresponding input pixel.  A region pair is denoted
($outcells$, $incells$).



\subsection{Lineage Modes}

\sys{} supports four modes of region lineage ({\it Full, Map, Pay, Comp}), and
one mode  of black-box lineage ({\it Blackbox}).  {\it cur\_modes} is set to
{\it Blackbox} when the operator does not need to generate any pairs (because
black box lineage is always in use).  {\it Full} lineage explicitly stores all
region pairs, and the other lineage modes reduce the amount of lineage that is
stored by partially computing lineage at query time using developer defined
mapping functions.  The following sections describe the modes in more detail.
%\srm{Is full lineage the same as cell lineage?  Confusing because in the
%previous section we said we supported cell lineage but here we don't talk about
%cell lineage at all.}

\subsubsection{Full Lineage}

Full lineage ({\it Full}) explicitly represents and stores all region pairs.
It is straightforward to instrument any operator to generate full lineage.
The developer simply writes code that generates region pairs and uses
$lwrite()$ to store the pairs.  For example, in the following CRD pseudocode,
if $cur\_modes$ contains {\it Full}, the code iterates through each cell in the
output, calculates the  lineage, and calls $lwrite()$ with lists of cell
coordinates.  Note that if {\it Full} is not specified, the operator can avoid
running the lineage related code.

{\footnotesize
\begin{alltt}
  def run(image, cur_modes):
     ...
     if \(Full \in\) cur_modes:
       for each cell in output:
         if cell == 1:
           neighs = get_neighbor_coords(cell)
           lwrite([cell.coord], neighs)
        else:
           lwrite([cell.coord], [cell.coord])
\end{alltt}
}

Although this lineage mode accurately records the lineage data, it is
potentially very expensive to both generate and store.  We have identified
several widely applicable operator properties that allow the operators to
generate more efficient modes of lineage, which we describe next.


\subsubsection{Mapping Lineage}

Mapping lineage ({\it Map}) compactly represents an operator's lineage
using a pair of mapping functions.  Many operators such as matrix transpose
exhibit a fixed execution structure that does not depend on the input cell
values.  These operators, called {\it mapping operators}, can compute forward
and backward lineage from a cell's coordinates and metadata (e.g., input and
output array sizes) and do not need to access array data values.  This is a
valuable property because mapping operators do not incur runtime and storage
overhead.  For example, one-to-one operators, such as matrix addition, are
mapping operators because an output cell only depends on the input cell at the
same coordinate, regardless of the value.  Developers implement a pair of
mapping functions, $map_f(cell, i)/map_b(cell, i)$, that calculate the
forward/backward lineage of an input/output cell's coordinates, with respect to
the $i$'th input array.  For example, a 2D transpose operator would implement
the following functions:

{\footnotesize
\begin{verbatim}
  def map_b((x,y), i):   def map_f((x,y), i):
     return [(y,x)]         return [(y,x)]
\end{verbatim}
}

Most SciDB operators (e.g., matrix multiply, join,
transpose,  convolution) are mapping operators, and we have
implemented their forward and backward mapping functions.  Mapping
operators in the astronomy and genomics benchmarks are depicted as
solid boxes (Figures \ref{f:lsstworkflow} and
\ref{f:genomicsworkflow}).

\subsubsection{Payload Lineage}

Rather than storing the input cells in each region pair, payload lineage
({\it Pay}) stores a small amount of data ({\it a payload}), and recomputes the
lineage using a payload-aware mapping function ($map_p()$). Unlike mapping
lineage,  the mapping function has access to the user-stored binary
payload. This mode is particularly useful when the operator has high fanin
and the payload is very small.  For example, suppose that the radius of
neighboring pixels that a cosmic ray pixel depends on increases with
brightness, then  payload lineage only stores the brightness insteall of the
input cell coordinates.   ({\it Payload operators}) call $lwrite(outcells,
payload)$ to pass in a list of output cell coordinates and a binary blob,
and define a {\it payload function}, $map_{p}(outcell, payload,
i)$, that directly computes the backward lineage of $outcell \in outcells$ from
the $outcell$ coordinate and the payload.  The result are input cells in the
$i$'th input array.  As with mapping functions, payload lineage does not need
to access array data values.  The following pseudocode stores radius values
instead of input cells:

{\footnotesize
\begin{alltt}
  def run(image,cur_modes):
     ...
     if \(PAY \in\) cur_modes:
       for each cell in output:
         if cell == 1:
           lwrite([cell.coord], '3')
         else:
           lwrite([cell.coord], '0')

  def map_p((x,y), payload, i):
     return get_neighbors((x,y), int(payload))
\end{alltt}
}

In the above implementation, each region pair stores the output cells and an
additional argument that represents the radius, as opposed to the neighboring
input cells.  When a backward lineage query is executed, \sys{} retrieves
the (outcells, payload) pairs that intersect with the query and executes
$map_p$ on each pair.  This approach is particularly powerful because the
payload can store arbitrary data -- anything from array data values to
lineage predicates~\cite{panda}.  Operators D to G in the two benchmarks
(Figures \ref{f:lsstworkflow} and \ref{f:genomicsworkflow}) are payload
operators.

Note that payload functions are designed to optimize execution of
backward lineage queries.  While \sys{} can index the input cells in
full lineage, the payload is a binary blob that cannot be easily indexed.
A forward query must iterate through each (outcells, payload) pair and
compute the input cells using $map_p$ before it can be compared to
the query coordinates.

\subsubsection{Composite Lineage}

Composite lineage ({\it Comp}) combines mapping and payload
lineage.  The mapping function defines the default relationship
between input and output cells, and results of the payload function
{\it overwrite} the default lineage if specified.    For example, CRD can
represent the default relationship -- each output cell depends on the
corresponding input cell in the same coordinate -- using a mapping
function, and write payload lineage for the cosmic ray
pixels:

{\footnotesize
\begin{alltt}
  def run(image,cur_modes):
    ...
    if \(COMP \in\) cur_modes:
       for each cell in output:
         if cell == 1:
          lwrite([cell.coord], 3)
      // else map_b defines default behavior

  def map_p((x,y), radius, i):
     return get_neighbors((x,y), radius)

  def map_b((x,y), i):
     return [(x,y)]
\end{alltt}
}

{\it Composite operators} can avoid storing lineage for a significant fraction
of the output cells.  Although it is similar to payload lineage in that the
payload cannot be indexed to optimize forward queries, the amount of payload
lineage that is stored may be small enough that iterating through the small
number of (outcells, payload) pairs is efficient.  Operators A,B and C in the
astronomy benchmark (Figure \ref{f:lsstworkflow}) are composite operators.

\subsection{Supporting Operator Re-execution}

An operator stores black-box lineage when $cur\_modes$ equals $Blackbox$.  When
\sys{} executes a lineage query on an operator that stored black-box lineage,
the operator is re-executed in tracing mode.   When the operator is re-run at
lineage query time, \sys{} passes $cur\_modes = Full$, which causes the
operator to perform $lwrite()$ calls.  The arguments to these calls are sent to
the query executor.

Rather than re-executing the operator on the full input arrays, \sys{} could
also reduce the size of the inputs by applying bounding box predicates prior to
re-execution.  The predicates would reduce both the amount of lineage that
needs to be stored and the amount of data that the operator needs to
re-process.  Although we extended both mapping and full operators to compute
and store bounding box predicates, we did not find it to be a widely useful
optimization.  During query execution, \sys{} must retrieve the bounding boxes
for every query cell, and either re-execute the operator for each box, or merge
the bounding boxes and re-run the operator using the merged predicate.
Unfortunately, the former approach incurs an overhead on each execution (to
read the input arrays and apply the predicates) that quickly becomes a significant
cost.  In the latter approach, the merged bounding box quickly expands to
encompass the full input array, which is equivalent to completely re-executing
the operator, but incurs the additional cost to retrieve the predicates.  For
these reasons, we do not further consider them here.

%performance of bounding box predicates in our
%experiments.


